\documentclass[11pt, answers]{exam}
\renewcommand{\baselinestretch}{1.05}
\usepackage{amsmath,amsthm,verbatim,amssymb,amsfonts,amscd, graphicx}
\usepackage{graphics}

\usepackage{afterpage}
\usepackage{caption}

\usepackage{tikz}
\usepackage{fancybox}

\usepackage{clrscode3e}

\topmargin0.0cm
\headheight0.0cm
\headsep0.0cm
\oddsidemargin0.0cm
\textheight23.0cm
\textwidth16.5cm
\footskip1.0cm
\theoremstyle{plain}
\newtheorem{theorem}{Theorem}
\newtheorem{corollary}{Corollary}
\newtheorem{lemma}{Lemma}
\newtheorem{proposition}{Proposition}
\newtheorem*{surfacecor}{Corollary 1}
\newtheorem{conjecture}{Conjecture}  
\theoremstyle{definition}
\newtheorem{definition}{Definition}

 \begin{document}
 


\title{CSC263: Assignment 2}
\date{February 9th, 2017}
\author{Junjie Cheng, Jiayun Liu, Zi Hao Lin}
\maketitle

\unframedsolutions

\begin{questions}
\question
%Question1
\begin{solution}
\begin{parts}
\part The auxiliary information is the following:

id: an int represent the id of the thread.

status: a character which is one of ${A, R, S}$, representing the status of the thread.

hasR: a Boolean indicating whether any node in the tree rooted at the current node represents a thread with status $R$.

\end{parts}
\end{solution}

\question
%Question2
\begin{solution}

\end{solution}


\question
%Question3
\begin{solution}
\begin{enumerate}
\item A Hash table of size $?$ and two arrays of length $26$ is used in the algorithm.

The assumption: SUHA


\end{enumerate}
\end{solution}

\end{questions}



\end{document}
