\documentclass[11pt, answers]{exam}
\renewcommand{\baselinestretch}{1.05}
\usepackage{amsmath,amsthm,verbatim,amssymb,amsfonts,amscd, graphicx}
\usepackage{graphics}

\usepackage{afterpage}
\usepackage{caption}

\usepackage{tikz}
\usepackage{fancybox}

\usepackage{clrscode3e}

\topmargin0.0cm
\headheight0.0cm
\headsep0.0cm
\oddsidemargin0.0cm
\textheight23.0cm
\textwidth16.5cm
\footskip1.0cm
\theoremstyle{plain}
\newtheorem{theorem}{Theorem}
\newtheorem{corollary}{Corollary}
\newtheorem{lemma}{Lemma}
\newtheorem{proposition}{Proposition}
\newtheorem*{surfacecor}{Corollary 1}
\newtheorem{conjecture}{Conjecture}  
\theoremstyle{definition}
\newtheorem{definition}{Definition}

 \begin{document}
 


\title{CSC263: Assignment 3}
\date{March 2nd, 2017}
\author{Junjie Cheng, Jiayun Liu, Zi Hao Lin}
\maketitle

\unframedsolutions

\begin{questions}
\question
%Question1
\begin{solution}
\begin{parts}
\part Each node will contain the following:
\begin{enumerate}
\item id: an int represent the id of the thread. This is the key.

\item parent: a pointer to the parent node.

\item left: a pointer to the left child.

\item right: a pointer to the right child.
\end{enumerate}

The auxiliary information is:
\begin{enumerate}
\item status: a character which is one of ${A, R, S}$, representing the status of the thread.
\item hasR: a Boolean indicating whether any node in the subtree rooted at the current node represents a thread with status $R$. 
\end{enumerate}

Sample tree:

The augmented data are labelled in the form (status\_hasR) beside each node.


\begin{minipage}[t]{\linewidth}
                \centering
\begin{tikzpicture}
[ 
level distance=9mm, 
every node/.style={circle, draw, inner sep=1.5pt},
level 1/.style={sibling distance=45mm}, 
level 2/.style={sibling distance=22mm}, 
level 3/.style={sibling distance=10mm},
level 4/.style={sibling distance=7mm},
label distance=-1mm
]

\node (root) [label=90:$A\_T$] {10} 
child {node[label=130:$S\_T$] {8} 
	child {node[label=130:$R\_T$] {7} 
		child {node[label=130:$A\_F$] {6}} 
		child {missing}
	}
	child {node[label=50:$R\_T$] {9}
	} 
}
child {node[label=50:$R\_T$] {14}
		child {node[label=130:$S\_F$] {12}
		    child{node[label=130:$A\_F$] {11}}
		    child{missing}
 	}
 	child {node[label=50:$A\_T$] {19}
		child {node[label=130:$R\_T$] {15}}
		child {node[label=50:$S\_F$] {21}}
	}
};
\end{tikzpicture}
\end{minipage}


\part 
\begin{enumerate}
    \item NewThread(t)
   
    First, create the node, then perform the AVL insert, which according to the AVL’s property will take O(logn).
    
     If the new thread's stat equals to 'R', during the path of insertion, update any bypassing node from False to True. All other nodes which are not on the path or nodes whose hasR is True remain unchanged. Since the update only add a few steps during insertion, it will still take O(logn). Moreover, during rotation, the only nodes need to be updated are the pivot nodes, so it takes O(1).

    \item Find(i)

    \item Completed(i)
    
    \item ChangeStatus(i, stat)
    \item ScheduleNext
\end{enumerate}

\end{parts}
\end{solution}

\question
%Question2
\begin{solution}
\begin{parts}
\part 
Since $h_1$ and $h_2$ are both under SUHA,  $prob(h_1(x) empty) = prob(h_1(x) empty) = \frac{k}{m} $ . x will not be inserted into an empty slot only if both $h_1(x)$ and $h_2(x)$ are not empty, otherwise x will choose the shorter chain, i.e. an empty slot. Therefore, the probability for x being inserted into an empty slot is $1-{(\frac{m-k}{m})}^2$.

\part 
$EX(length) = 3 \times \frac{7}{16} +  6 \times \frac{5}{16} +  9 \times \frac{4}{16} = \frac{87}{16}$

Verification: Counting each possibility and resulted length in the table, we could get the formula above.\\

\centering
\begin{tabular}{ccccc}
$h_1(x)$/$h_2(x)$    & T{[}0{]} & T{[}1{]} & T{[}2{]} & T{[}3{]} \\
T{[}0{]} & 6        & 3        & 6        & 6        \\
T{[}1{]} & 3        & 3        & 3        & 3        \\
T{[}2{]} & 6        & 3        & 9        & 9        \\
T{[}3{]} & 6        & 3        & 9        & 9       
\end{tabular}




\end{parts}
\end{solution}


\question
%Question3
\begin{solution}
\begin{enumerate}
\item A Hash table of size $m$, containing elements with the attributes (key: String, freq: int), an array of String and an array of int are used in the algorithm. Both arrays have length of $26$.

The assumption: SUHA

(Other assumptions)
We assume $\proc{$Chained\_Hash\_Search$}(T, key)$ returns null if there is no element with that key in the hash table.

We also assume the array index begins with 0.

We also assume that all words inputs begins with English alphabet a-z and A-Z.

\item There are $2$ arrays of length $26$, $A$ and $B$, to keep track of the most frequent word that begins with each alphabet and the frequency of that word. (The first element will represent words that begin with letter 'a', the second will represent words that begin with letter 'b', etc.) Initially, each element in $A$ is $null$ and each element in $B$ is $0$. 

When a word $w$ (other than query) is input, we first search whether there is an element $x$ with $x.key=w$. If $x$ does not exist then we create the element $y$ with $y.key=w$ and $y.freq=1$ and hash it into the hash table $T$. If $x$ exists, then we increment $x.freq$ by $1$.

After the element is created or updated, we extract the first character of the word and compare the attribute $freq$ with the corresponding highest frequency (among the words beginning with the same alphabet). If the current word has a higher frequency (or has the same frequency but occurs earlier in alphabetical order), then we update the arrays $A$ and $B$ with the new word and new frequency. 

When a query occurs, the algorithm will return the words in array $A$ one by one, i.e. the most frequent word beginning with each letter in sorted order.

\item 
The following variables are used by both code to process an input word and the code for query.
\begin{codebox}
\li T // The hash table
\li A // The array keeping track of the most frequent words beginning with each letter
\li B // The array keeping track of the corresponding frequency
\end{codebox}
The code to process an input word.
We assume $\proc{Chained\_Hash\_Search}(T, key)$ returns null if there is no element with that key in the hash table.
We also assume the array index begins with 0.
\begin{codebox}
\Procname{$\proc{Process\_word}(word, T, A, B)$}
\li $x \gets \proc{Chained\_Hash\_Search}(T, word)$ 
\li \If $(x)$ // The element exists; the word has been input before
\li 	\Then $x.freq++$; 
\li \Else // The word has not been input before
\li 	$y.key=word$; $y.freq=1$; // Create a object y with key=word and freq=1
\li 	$\proc{Chained\_Hash\_Insert}(T, y)$;
\li 	$x\gets y$ \End // Use variable x to represent the element
\li $c \gets word[0].lowercase$; // Get the lower case of the first letter of the input word.
\li $d \gets c-'a'$ // Arithmetic operations on chars' ASCII value. Note 'a' is a char.
\li \If $(x.freq > B[d]$ or $(x.freq=B[d]$ and $word<A[d]))$ 
\li     \Then $A[d] \gets word$;
\li     $B[d] \gets x.freq$; \End
\end{codebox}

\begin{codebox}
\Procname{$\proc{Process\_query}(T, A, B)$}
\li \For $(i=0; i<26; i++)$ \Do
\li     $\proc{$print$}(A[i])$;
\end{codebox}

\item
Input process time:

Under the assumption of simple uniform hashing, by theorem 11.1 and theorem 11.2 in CLRS, $\proc{Chained\_Hash\_Search}(T, key)$ takes $\Theta{(1+\alpha)}$ time, regardless of whether it is successful, where, $\alpha=\frac{l}{m}$ is a constant. So, $\proc{Chained\_Hash\_Search}(T, key)$ takes $\Theta{(1+\alpha)}=\Theta{(1+\frac{l}{m})}=\Theta{(1)}$. 

$\proc{Chained\_Hash\_Insert}(T, y)$, according to the CLRS book, has worst-case running time $O(1)$, which indicates that it is also in $\Theta{(1)}$. So the expected running time for line 6 is also $\Theta{(1)}$.

Also note that all other lines in $\proc{Process\_word}(word, T, A, B)$ consists only finite number of comparisons, assignments, and arithmetic operations, which are all $\Theta{(1)}$ operations. So the expected running time for all other lines is also $\Theta{(1)}$.

Since all parts of the code has expected running time $\Theta{(1)}$ , the expected time to process each input is $\Theta{(1)}$.

Query operation perform time:

Consider line 1 of $\proc{Process\_query}(T, A, B)$, in the worst case, there are 27 assignments and 27 comparisons made. In the loop, print operation in line 2 is executed 26 times. Since assignment, comparison and print all take linear time, the worst-case time complexity for query operation is $\Theta{(1)}$.
\end{enumerate}
\end{solution}

\end{questions}
\end{document}
